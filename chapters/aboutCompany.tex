\chapter{Байгууллагын тухай}

\hspace{0.5cm}
"MOGUL" групп нь 1997 онд компьютер, компьютерийн тоног төхөөрөмж нэвтрүүлэх, үйлчилгээ үзүүлэх зорилгоор үйл ажиллагаагаа эхэлж байсан бөгөөд Мэдээллийн Технологийн чиглэлээр төрөлжин үйл ажиллагаа явуулдаг 6 компанитайгаар 26 дахь жилдээ ажиллаж байгаа. Нийт 380 гаруй ажилтан, тэдгээрийн 200 гаруй ажилтан нь инженерүүд байдаг. Мэдээллийн технологийн дэд бүтэц, тоног төхөөрөмжийн худалдаа, үйлчилгээ, Мэдээллийн болон биет аюулгүй байдал, Программ хангамж үйлдвэрлэл, Цахим засаг, Клауд болон Менежед үйлчилгээ, Дата болон AI, салбаруудын мэдээллийн технологийн шийдэл чиглэлээр үйл ажиллагаа явуулдаг.

\section{Компаний тухай}
"MOGUL" групп нь 2023 оны өвлийн улиралд 6 салбар компанитай байсан бөгөөд эдгээрт: 
\begin{itemize}
	\item ITZone ХХК
	\item Новелсофт ХХК
	\item Могул Сервис энд Саппорт ХХК
	\item Дижитал Воркс ХХК
	\item Дижитал Повер ХХК
	\item Могул Экспресс ХХК
\end{itemize}
компаниуд орно. "Новелсофт" ХХК-ийн бизнесийн үйл ажиллагаа өргөжсөнөөр "Nomadic Software Solution" ХХК компанийг 2023 оны хаврын улирал үүсгэн байгуулсан. 

\section{Үйл ажиллагаа}
"Новелсофт" ХХК нь захиалгат програм хангамж хөгжүүлэлт, дата аналитик, дата менежмент, мэргэжлийн үйлчилгээ (outsourcing) зэрэг үйл ажиллагаа явуулдаг байсан ба үүнээс програм хангамж болон дата аналитик гэх үйл ажиллагааны хүрээнд 2 хуваагдан "Nomadic SS" ХХК бий болсон. 

"Nomadic SS" ХХК нь одоогоор програм хангамж хөгжүүлэлт, мэргэжлийн үйлчилгээ (outsourcing) зэрэг үйл ажиллагааг явуулдаг.

\section{Технологиуд болон системүүд}
Компаний хувьд хөгжүүлэгдэж буй системүүд нь цогц системүүд байдаг ба дийлэнх хөгжүүлэгчдийн туршлага, системийн зохиомжоос хамааран
\begin{itemize}
	\item Front-end
	\begin{itemize}
		\item Mobile: Flutter - BLoC (Business Logic Components)
		\item Web: Angular - Primeng
		\item Desktop: .NET
	\end{itemize}
	\item Server
	\begin{itemize}
		\item Java - JHipster Spring boot
		\item OAuth - Keycloak
		\item Cassandra
		\item Kafka
		\item Redis
		\item Prometheus
	\end{itemize}
\end{itemize} 
технологиудыг best practice болгон мөрдөж ашигладаг. Хэрэглэгчийн шаардлагаас үүдэн өөр технологиудыг ашигласан тохиолдлууд ч байдаг.